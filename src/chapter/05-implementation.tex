% !TEX root = ../report.tex

\chapter{Implementation}

\minitoc

This chapter discusses the major requirements, and implementation of the different algorithms and datastructures in the system. And is meant for the reader to get insight into what was actually implemented.

TODO: Something about each section after done
\clearpage

\section{Major Requirements}\label{impl:Major Requirements}
\subsection{FR1}
\begin{quotation}
\em The system must be able to harvest tweets and/or users from Twitter that are related to a movie in the Netflix-dataset %not to edit, unless edit for all
\end{quotation}

\subsection{FR6}
\begin{quotation}
\em The system must be able to supplement Netflix-dataset with tweets and/or users from Twitter that are related to a movie in the Netflix-dataset %not to edit, unless edit for all
\end{quotation}

\subsection{FR7}
\begin{quotation}
\em The system must be able to predict ratings of the movies for all users in any dataset that reflects the Netflix dataset %not to edit, unless edit for all
\end{quotation}

Since the data from twitter is not gathered and combined with the Netflix-data this requirement is not implemented, but rather a task for further work.

\subsection{NFR1}
\begin{quotation}
\em The system must be able to cost efficiently predict a rating of a movie for a given user %not to edit, unless edit for all
\end{quotation}



\section{Data structures}\label{impl:Data structures}
algoritmeneog datastrukturene,

\section{Functional Modules}\label{impl:Functional Modules}
fremhev noen nye/originale funksjoner.
\subsection{FunctionalFields}
\subsection{Harvesters}

\section{Testing}\label{impl:Testing}
Også diskuter hvordan du har tenkt å utføre testen din (validering og evaluering).

% !TEX root = ../report.tex

\chapter{Implementation}

\minitoc

I implementasjonskapittelet diskuter de viktigste algoritmene
og datastrukturene, og hvordan de utviklet seg, fremhev noen
nye/originale funksjoner. Også diskuter hvordan du har
tenkt å utføre testen din (validering og evaluering).

\clearpage

\section{The moivie-suggestion algorithm}
% About Classification
% http://www.netflixprize.com/assets/GrandPrize2009_BPC_PragmaticTheory.pdf
% http://www.netflixprize.com/assets/GrandPrize2009_BPC_BellKor.pdf
% http://www.netflixprize.com/assets/GrandPrize2009_BPC_BigChaos.pdf
% http://www2.research.att.com/~volinsky/netflix/Bellkor2008.pdf
% http://public.research.att.com/~volinsky/netflix/cfworkshop.pdf
% http://public.research.att.com/~volinsky/netflix/BellKorICDM07.pdf
% http://www2.research.att.com/~volinsky/netflix/ProgressPrize2007BellKorSolution.pdf
% http://public.research.att.com/~volinsky/netflix/sigkddexp.pdf
% http://www.netflixprize.com//leaderboard
% https://sites.google.com/site/xlvector/netflixprize/papers
% http://www.slideshare.net/bmabey/svd-and-the-netflix-dataset-presentation
% http://lenskit.grouplens.org/

% https://github.com/cloudera/oryx

How to best make something out of the dataset
How to produce the RMSE
What is the RMSE


TODO:
There are two basic problems when analysing very large datasets:
    - Writing the code correctly.
    - The performance (i.e. speed at which it runs).



Something about why and which classification alg to use
% http://scenic.princeton.edu/network20q/wiki/index.php?title=Q4:_How_does_Netflix_recommend_movies%3F
% http://blog.jimjh.com/static/downloads/2013/05/12/netflix.pdf
There were 51 051 contestants on 41 305 different teams competing in the Netflix-Prize competition, and a lot of these contestants shared their solution through code repositories such as GitHub (--------------something about github here?--------------). Since many of these solution scored very well on the contest it was natural to explore these and use their solution to test potential improvement in collaboration with the social media data harvested from Twitter.

Here we explore some of the better solutions and how and why these could be used to our benefit.
Here we explore some of the better solutions and put together how these can be used to


The moivie-suggestion algorithm:


\subsection{Using the Netflix-data}
The Netflix-data has, as mentioned in the section about the dataset \ref{subsec:netflixdata}, a set of ratings from users for each movie, and a time-stamp for these ratings. As vi saw from the study about the winning contestants \ref{subsec:thewinners}, we saw that this information alone let them produce a system which could predict movie-ratings with a RMSE of 0.8567. They saw

\subsection{Using the Twitter-data}

\subsection{Conclusion}


\section{Covered Requirements}

\section{Data integration}

\section{Testing}

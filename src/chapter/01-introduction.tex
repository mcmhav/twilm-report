% !TEX root = ../report.tex

\chapter{Introduction}

\minitoc
\setcounter{page}{1}
\pagenumbering{arabic}

This chapter will give an overview of what this project is about, and give the purpose and motivation behind this project. It will also outline the project for easy chapter understanding, and let the reader know where different parts of the project is located in this paper. The paper serves as a documentation of the work done.

\clearpage


\section{Purpose}
The goal of this project is to explore the correlation between social media and movie recommendation, and put down the cornerstones for such a system. The system is intended to produce ratings for movies based on a social media network (Twitter) and a set of movies with movie-ratings (Netflix-dataset). The social media network will be harvested for movies based on the movies from the movie-dataset, then these two datasets will be combined to open the possibility of TODO: maybe do this one together btw :P

TODO:fr&m klekk out the goals
\subsubsection{Goals}
\begin{enumerate}
  \item Produce well documented reasons to why some directions are preferred before other when dealing with social media and movie recommendation.
  \item Create a working harvester to gather needed data from social media.
  \item Investigate different algorithms to predict ratings on the extended dataset.
\end{enumerate}

The report is the cornerstone of this system, and is meant to ease further development and explain reasons to why some directions are to be preferred before other directions.


\section{Motivation}
Social media networks are well established networks, they possess and produce big amounts of data on a daily basis. On Twitter, a social media site, 9 100 messages are posted every second~\cite{twitt-stats}, and on Facebook, another social media site, 1 million links are shared every 20 minuets~\cite{face-stats}. With this amount of data produced conclusions and predictions can be made through gathering the data and performing collaborative filtering on this data. One of the potential predictions to draw from this is movie ratings predictions. Hence the motivation for utilizing the vast amount of data for movie rating predictions.


\section{Context}
The coming chapters describes the course of the project in order of the structure. And how social media can be used in correlation with movie recommendation. Table~\ref{table-reportstructure} gives a short insight to the content of the different chapters.

\begin{table}
\centering
\begin{tabularx}{\textwidth}{ l X l }
  \textbf{Chapter}      & \textbf{Description} \\
  \hline \\ [-1.5ex]
  Chapter 1 & The Introduction chapter gives an overview of the project to the reader, and explains the tasks the project is meant to solve. \\
  \hline \\ [-1.5ex]
  Chapter 2 & The Preliminary Study chapter describes work and research done, and how and why the choices made were made. This includes information about different technologies, such as Netflix and Twitter, how they can be used, different algorithms to work with and similar solutions. \\
  \hline \\ [-1.5ex]
  Chapter 3 & The Requirements chapter describes how the requirements where acquired and the major functional and non functional requirements of the system and why these are important for the system. \\
  \hline \\ [-1.5ex]
  Chapter 4 & The Design chapter explores how the system is structured, and describes the different parts of the system and how they correlates. \\
  \hline \\ [-1.5ex]
  Chapter 5 & The Implementation chapter explores in detail the implementation of gathering of twitter information. TODO: something more?  \\
  \hline \\ [-1.5ex]
  Chapter 6 & Evaluation chapter includes how the research went, what to take from the results drawn from the research done and issues during the project. \\
  \hline \\ [-1.5ex]
  Chapter 7 & The Conclusion chapter sums up the project and describes the findings and reflections around this. It also explores further work to be done. \\
  \hline \\ [-1.5ex]
  Appendix & The appendix contains extended information such as a full list of the requirements and in depth design. \\
\end{tabularx}
\caption{Structure of Report}
\label{table-reportstructure}
\end{table}

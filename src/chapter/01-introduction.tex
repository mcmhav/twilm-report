\chapter{Introduction}

\minitoc

Context;  motivation  for the project;  problem statement;  outline of
dissertation.

I introduksjonskapittelet, forklar kort
konteksten med og motivasjonen bak arbeidet
ditt, forklar problemet som du prøver å løse og
forklar hvorfor dette problemet er verdt å løse.


Wonsole is a student project under the TDT4290 course in IDI, NTNU. The project is intended to give all its students experience in a customer guided IT- project and the feel of managing a project in a group. Every phase of a typical IT- project will be covered. This report will serve as documentation of our work. This includes our work progress, the technologies we used, our research findings and so on. The introduction chapter is meant to describe the project, our goals and briefly the involved parties.

\clearpage


\section{Purpose}
\section{Motivation}
\section{Context}

\section{Project}
This is a proof of concept project. The underlying task is to research and develop a system where power users can benefit from a console.  The concept aims to ease the workload of a power user who is working with object editing, and to see how the efficiency of a console might prove to improve the work. The power user is usually a user who often works with the system over a longer time, and is in depth familiar with the system. We will research already existing systems of this kind, and look at the possibilities and advantages of such a system in a chosen domain.

We have chosen a library as our domain, and this will be used to explore and test the concept. The library domain is chosen since it possesses potential for the existence of power users and multiple input forms which could be made more efficient through a console. This domain also opens the opportunity to test our system on for instance employees on campus, which is important for the proving of the concept.
\subsubsection{Goals}
\begin{enumerate}
  \item Provide extensive documentation and a successful presentation of the end product.
  \item Create a working prototype of a system where a scripting console is embedded into a modern web interface. The console should provide access to viewing and modifying the underlying data objects of the system's domain via a Domain Specific Language(DSL).
  \item Investigate the ramifications of the added functionality, in terms of usability and technical aspects.
\end{enumerate}

It is important to note that the report is in focus. It will be the cornerstone of the prototype, to not just ease further development, but also to amplify the reasons for the choices we make.



\section{Project name}
Project name is important project identificator. It should summarize main project goal or functionality. In real projects, this is often a trademark, or a name that reflects the name of the company. For our project, the main concern was to create a descriptor that reflected the root concept, namely incorporating a console into a web application.

We held a brainstorming meeting specifically to create a name for the project. Early in the process, we created a list of words that could describe our project functionality or goal. Some keywords:
\begin{itemize}
\item Master, Super User
\item Console, Terminal, Command Line
\item Web Application, GUI
\item Internet, Networking
\item Text, Keyboard
\end{itemize}

From these keywords we attempted to compile a list of candidate names:
\begin{itemize}
\item Console 2.0
\item Wonsole
\item Wensole
\item Websole
\item Werminal
\item interCLI
\end{itemize}

After a discussion and a brief investigation into which names were already taken by other projects, we chose the name \emph{Wonsole}. The project name alone can be a little confusing, so we added the subtitle: \emph{The new web console for power users.}

\section{Structure of Report}
This report describes the development of bringing the scripting experience to the power user.
The report is structured after the course of the project, and gives the reader insight into the development of the system.

\begin{table}
\centering
\begin{tabularx}{\textwidth}{ l X l }
  \textbf{Chapter}      & \textbf{Description} \\
  \hline \\ [-1.5ex]
  Chapter 1   & The introduction chapter introduces the problem to the reader, introduce the members and stakeholders of this project and explain the motivation for doing this project. After reading this chapter, the reader will be left with an overall view of the projects outline and goals. \vspace*{0.7ex} \\
  \hline \\ [-1.5ex]
  Chapter 2                   & The Preliminary Study chapter describes the work and research done, and road taken to the choices we made. This includes technologies, methodologies, testing and version control. \\
  \hline \\ [-1.5ex]
  Chapter 3          & The Project management chapter deals with how the team will work to reach the desired goal. This includes the structure of the team, the plan for the project, constraints the team and project is put under and quality assurance. \\
  \hline \\ [-1.5ex]
  Chapter 4   & The Requirements chapter describes the backlog and rationale for this. Use cases for the system is also placed in this chapter together with user stories. \\
  \hline \\ [-1.5ex]
  Chapter 5                    & The Test Plan chapter contains the approach for testing with the overall test plan for the project, and the test scheduling. \\
  \hline \\ [-1.5ex]
  Chapter 6              & The Architecture chapter explains the structure of the system, how it is put together and why it is put together the way it is. \\
  \hline \\ [-1.5ex]
  Chapter 7 - 10              & In the sprint chapters the reader can see how the product has developed during the time of the project. This includes planning, architecture, the implementation, testing and evaluation of the sprint. \\
  \hline \\ [-1.5ex]
  Chapter 11              & Evaluation chapter includes the team's view on the team dynamics, relationship with the customer, issues during the project, the planning and the quality assurance. \\
  \hline \\ [-1.5ex]
  Chapter 12              & Conclusion chapter sums up the project and describes the findings and reflections around this. \\
  \hline \\ [-1.5ex]
  Appendix              & The appendix contains tables, templates, the test cases and more. \\
\end{tabularx}
\caption{Structure of Report}
\label{table-reportstructure}
\end{table}

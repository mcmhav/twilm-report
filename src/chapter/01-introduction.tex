% !TEX root = ../report.tex

\chapter{Introduction}

\minitoc
\setcounter{page}{1}
\pagenumbering{arabic}

This chapter will give an overview of what the project is about, and the purpose and motivation behind the project. It will outline the different chapters, and make it clear to the reader where different parts of the project is located in the paper. The paper serves as a documentation of the work done.

\clearpage


\section{Purpose}
Social media data is used frequently in recommendation systems. There are many ways in which this can be done, both in terms of what data is used and what is recommended. The purpose of the project is to explore one of these possibilities, namely how data from Twitter can be used to predict movie ratings and recommend movies. Also, the purpose is to design a system that can use data from Twitter in such a way. Furthermore, the report is meant to ease further development of such a system and explain reasons to why some directions are preferred over other directions.

\renewcommand{\arraystretch}{1.5}
\begin{table}[H]
    \centering
    \begin{tabular}{ p{10cm} }
        \textbf{Goals} \\ \hline
        Understand how data from Twitter can be used for recommendations and predictions. \\
        Implement harvesting of data from Twitter. \\
        Gather data from Twitter. \\
        Supplement the Netflix Prize dataset with data from Twitter. \\
        Implement a Netflix Prize movie rating prediction system. \\
        Compare RMSE scores. \\
    \end{tabular}
    \captionof{table}[Main goals]{The main goals of the research}
    \label{tab:main-goals}
\end{table}
\renewcommand{\arraystretch}{1}

\section{Motivation}
Social media networks are well established systems. They harbor and produce huge amounts of data on a daily basis. On Twitter, a social media network, 9 100 messages are posted every second~\cite{twitt-stats}, and on Facebook, another social media network, 1 million links are shared every 20 minuets~\cite{face-stats}. Predictions and recommendations based on this data can be done by gathering some of it and running information retrieval algorithms on it. Several of these social media networks have APIs and ways of making the data they produce accessible to developers. One of the potential ways this data can be used is for movie rating predictions. Hence the motivation for utilizing this huge amount of data for movie rating predictions.

\section{Context}
The following section describes how the project is structured. Table~\ref{table-reportstructure} gives a brief overview of the content of the different chapters in the report.

\begin{table}
\centering
\begin{tabularx}{\textwidth}{ l X l }
  \textbf{Chapter}      & \textbf{Description} \\
  \hline \\ [-1.5ex]
  Chapter 1 & The Introduction chapter gives an overview of the project to the reader. It also outlines the purpose and motivation of the project. \\
  \hline \\ [-1.5ex]
  Chapter 2 & The Preliminary Study chapter documents knowledge, research and technology that is relevant to the project, and how and why some of them were prioritized over others when it comes to how they are used in the project. Twitter and the Netflix Prize dataset is documented in this chapter. \\
  \hline \\ [-1.5ex]
  Chapter 3 & The Requirements chapter describes the requirements of the project. It also describes how and why they were created. \\
  \hline \\ [-1.5ex]
  Chapter 4 & The Design chapter describes the design of the system and how it was made. \\
  \hline \\ [-1.5ex]
  Chapter 5 & The Implementation chapter describes the implementation of the system. Mainly, the implementation of the system harvesting data from Twitter is described. \\
  \hline \\ [-1.5ex]
  Chapter 6 & Evaluation chapter discussed the development process, testing of results and major issues. \\
  \hline \\ [-1.5ex]
  Chapter 7 & The Conclusion chapter sums up the project and describes the findings and reflects on them. It also describes further work to be done. \\
  \hline \\ [-1.5ex]
  Appendix & The appendix contains extended information such as a full list of the requirements. \\
\end{tabularx}
\caption{Structure and chapters of the report.}
\label{table-reportstructure}
\end{table}

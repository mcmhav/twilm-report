% !TEX root = ../report.tex

\chapter{Requirements}

\minitoc

How requirements  were  captured;
  discussion  of  major requirements
(referring  to  Appendix  A for details).

I “Requirement”-kapittelet forklar hvordan du satt opp kravene.
Men ikke ha med en fullstendig oversikt over kravene her!


\clearpage

\section{Functional Requirements}
This section will look at how the functional requirements where gathered, and look at some of the more central ones. For a full list of the functional requirements see appendix~\ref{app:req}.

Based on the findings in the prestudy and the initial task at hand, the requirements had a basis to get formed on. Prestudy on the similar solutions opened

\begin{itemize}
  \item The system must be able to harvest tweets and/or users from Twitter that are related to a movie in the netflix-dataset.
  \item The system must be able to harvest tweet.user from Twitter
  \item The system must be able to harvest user.followees from Twitter
  \item The system must be able to harvest user.followers from Twitter
  \item The system must be able to harvest user.tweets from Twitter

  \item The system must be able to supplement netflix-dataset with tweets and/or users from Twitter that are related to a movie in the netflix-dataset.
  \item The system must be able to test recommendation based on the netflix dataset
  \item The system must be able to test recommendation based on the netflix-twitter-mixed dataset
  \item The system must be able to track the progress of harvesting
  \item The system must be able to track the progress of data-building
  \item The system must be able to track the progress of prediction
  \item The system must be able to give user-based recommendation
  \item
\end{itemize}

\section{Non Functional Requirements}

\begin{itemize}
  \item System should improve the RMSE
\end{itemize}


\section{Requirements Evaluation}


\section{Prioritized Requirements}


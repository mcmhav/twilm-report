% !TEX root = ../report.tex

\chapter{Requirements}

\minitoc

This chapter will describe the capturing of the requirements and go into depth on the most central ones.

\clearpage

\section{Capturing the Requirements}
Based on the initial task at hand and the findings in the prestudy, the requirements had a basis to get formed on. The initial task gave an overview of how the requirements of the system should look like. The prestudy on similar solutions made it clearer what was feasible to do and what would be impractical to do. For instance runtime of different recommendation algorithms and their ability to be parallelized. The prestudy on Twitter made it clear how Twitter can be explored and how it can be done in a timely manner. It also showed the difficulties in gathering sufficient data from the APIs and that scraping would work but could not be done due to legal considerations.

\section{Functional Requirements}\label{section:functional-requirements}
This section will look at the major functional requirements, and see why these are central to the system. For a full list of the functional requirements see appendix~\ref{app:req}.
\begin{description}
  \item[FR1] The system must be able to harvest tweets and/or users from Twitter that are related to a movie in the Netflix Prize dataset
  \item[FR6] The system must be able to supplement Netflix Prize dataset with tweets and/or users from Twitter that are related to a movie in the Netflix Prize dataset
  \item[FR7] The system must be able to predict ratings of the movies for all users in any dataset that reflects the Netflix Prize dataset
\end{description}

\subsubsection{FR1}
Data from Twitter is essential for any recommendation to be done, so the system must be able to harvest data from Twitter. The harvested data must be related to the the movies existing in the Netflix dataset. This is because the amount of available data is vast. Unnecessary data will just produce static to the dataset, which is bad for the prediction system.

\subsubsection{FR6}
The Netflix Prize dataset consists of a set of movies and a set of users with ratings to these movies, this set is quite sparse as explored in~\ref{subsec:netflixdata}. By harvesting tweets from Twitter this level of sparseness is expected to be reduced, and therefore increase the accuracy of the data. With a more complete set of data, and less sparseness, the algorithms to produce prediction of ratings will perform better~\cite{grobelnikDataSparsityIssues}. So the system must be able to merge the data from the Netflix Prize and the harvested data from Twitter, so predictions can be made on the complete dataset.

\subsubsection{FR7}
The system is expected to make predictions on user ratings of movies. There will be made different datasets, at least one of the Netflix Prize dataset alone, and one where the Netflix Prize dataset is supplemented with tweets. It is important that the system is able to make predictions on the sets so comparisons and evaluations of the predictions can be made.
With this comes the importance of predicting ratings of movies for all the users. It is not just one user who can get movie recommendations. The dataset will be ever growing, as new tweets will be added to the database continuously, the system must therefore be able to do prediction on the changing dataset.

\section{Non Functional Requirements}\label{section:non-functional-requirements}
This section will look into the major non functional requirement, and see why this is central to the system. For a full list of the non functional requirements see appendix~\ref{app:req}.
\begin{description}
  \item[NFR1] The system must be able to cost efficiently predict a rating of a movie for a given user
\end{description}

\subsubsection{NFR1}
The main concern with the efficiency of the system is that if it takes too much time to calculate a prediction then the prediction might end up being of no help to the user. For instance if a user is rating a movie highly, and the rating of this movie will lead to higher certainty predictions for other movies, then the system must be able to efficiently apply this new information to the predictions. And since the dataset will be big is it important that it does predictions efficiently.


\section{Prioritized Requirements}
The order of the major functional requirements from section~\ref{section:functional-requirements} is the prioritized order also. \textbf{FR1} and \textbf{FR6} must be in the order they are numbered since the next will not work without the one before, meaning, the harvested data is needed to be able to supplement the Netflix data with Twitter data.

The \textbf{NFR1} is after this, since it is important that the system has a good response time, but this will not matter without the actual data.

\textbf{FR7} comes after this, since it is necessary to have a merged Netflix Twitter dataset to be able to do any collaborative filtering on the data.

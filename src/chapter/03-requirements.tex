% !TEX root = ../report.tex

\chapter{Requirements}

\minitoc

How requirements  were  captured;
  discussion  of  major requirements
(referring  to  Appendix  A for details).

I “Requirement”-kapittelet forklar hvordan du satt opp kravene.
Men ikke ha med en fullstendig oversikt over kravene her!


\clearpage

\section{Capturing the Requirements}
Based on the initial task at hand and the findings in the prestudy and, the requirements had a basis to get formed on. The initial task gave a coarse grained overview of how the requirements of the system should look like. The prestudy on similar solutions made what was feasible to do and what would be impractical to do clearer. For instance runtime of different recommendation algorithms and their ability to be parallelized. The prestudy on Twitter made it clear how Twitter can be explored and how it can be done in a timely manner.

TODO: something about how twitter's api is slow, scraping, 2 lines.
An interesting find was that most of the recommendation systems for the Netflix prize had a more stationary ground to them, they only tried to get the RMSE as low as possible, not considering the fact that ratings would be constantly updated, and therefore a requirement to do live recommending with new input-information would require the already heavy computations to be run constantly on all the data often, which is less than feasible.


\section{Functional Requirements}
This section will look at how the functional requirements where gathered, and look at some of the more central ones. For a full list of the functional requirements see appendix~\ref{app:req}.


\begin{itemize}
  \item The system must be able to harvest tweets and/or users from Twitter that are related to a movie in the netflix-dataset
  \item The system must be able to harvest tweet.user from Twitter
  \item The system must be able to harvest user.followees from Twitter
  \item The system must be able to harvest user.followers from Twitter
  \item The system must be able to harvest user.tweets from Twitter

  \item The system must be able to supplement netflix-dataset with tweets and/or users from Twitter that are related to a movie in the netflix-dataset
  \item The system must be able to learn recommendations based on the netflix-twitter-mixed dataset
  \item The system must be able to produce a test-dataset
  \item The system must be able to test recommendation based on the netflix dataset and produce a RMSE
  \item The system must be able to test recommendation based on the netflix-twitter-mixed dataset and produce a RMSE
  \item The system must be able to track the progress of harvesting
  \item The system must be able to track the progress of data-building
  \item The system must be able to track the progress of prediction
  \item The system must be able to give user-based recommendation
  \item The system must be able to
  \item The system must be able to
  \item The system must be able to
  \item The system must be able to
  \item The system must be able to
  \item The system must be able to
\end{itemize}

\section{Non Functional Requirements}

\begin{itemize}
  \item System should improve the RMSE
\end{itemize}


\section{Requirements Evaluation}


\section{Prioritized Requirements}


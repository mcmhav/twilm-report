% !TEX root = ../report.tex

\chapter{Requirements}

\minitoc

How requirements  were  captured;
  discussion  of  major requirements
(referring  to  Appendix  A for details).

I “Requirement”-kapittelet forklar hvordan du satt opp kravene.
Men ikke ha med en fullstendig oversikt over kravene her!


\clearpage

\section{Functional Requirements}
This section will look at how the functional requirements where gathered, and look at some of the more central ones. For a full list of the functional requirements see appendix~\ref{app:req}.

Based on the findings from the prestudy and the initial task at hand the requirements had a basis to get formed on. Prestudy on the similar solutions opened

\begin{itemize}
  \item The system must be able to access Twitter-tweets
  \item The system must be able to harvest a Twitter-user for tweets
  \item The system must be able to find movies in user-tweets
  \item The system must be able to find movie-tweets from the tweeters a user is following

  \item The system must be able to supplement netflix-dataset with movie-tweets from Twitter
  \item The system must be able to test recommendation based on the netflix dataset
  \item The system must be able to test recommendation based on the netflix-twitter-mixed dataset
  \item The system must be able to track the progress of harvesting
  \item The system must be able to track the progress of data-building
  \item The system must be able to track the progress of prediction
  \item The system must be able to give user-based recommendation
  \item
\end{itemize}

\section{Non Functional Requirements}

\begin{itemize}
  \item System should improve the RMSE
\end{itemize}


\section{Requirements Evaluation}


\section{Prioritized Requirements}


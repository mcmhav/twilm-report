% !TEX root = ../report.tex

\chapter{Conclusion}

\minitoc

This chapter will describe the final version of our product and what we have found during this project. Also we will discuss the further development of the system and what improvement can be made.

\clearpage

\section{Final Product}
STATUS
TODO

\subsection{Chosen Technologies}
We spent a lot of time on the pre- study and on research in this project, and we feel it was well worth the effort. Also our customer helped us a lot, and offered great suggestions and guidance throughout the project. The technologies we ended up using were not only new and exciting, but were also a great fit together. JavaScript, jQuery, JSON, HTML and CSS are robust and mature web technologies in use all over the world today. Coupled with CouchDB, which was designed from the ground with web- applications in mind, they make a powerful collection of technologies that work well together.

Probably the most important decision on technology we were left with after the pre- study was whether to go for a traditional database management system like MySQL, or choose a more unconventional NoSQL system. The customer did not specify any requirements concerning the performance or stability of the server in this project, and indicated that the specifics on the server implementation was not that important to them. As a result we wanted it to be as simple as possible. After a detailed look at the needs for this project we ended up choosing a NoSQL database in the form of CouchDB. The reason for this was that CouchDB provided us with many advantages that a traditional relational database could not. Our system is heavily centered around JavaScript objects and their attributes. CouchDB, and other document oriented NoSQL systems, stores data in the form of objects directly in JSON format, which is a simple textual notation for JavaScript objects. This implied a minimal amount of work getting the client and server to communicate. If we had opted to use a traditional SQL system for the database we would have needed more extensive server side technologies, like PHP or ASP.NET to handle the database and to do database-object parsing. All of this was avoided by choosing CouchDB, which is able to work without any extra server software, and automatically creates a RESTful api to interact with the database.



\section{Findings}
TODO
Oppsummer hva du har oppnådd,
sammenlignet med hva du opprinnelig ønsket å oppnå.

\section{Related Work}
TODO
Relater arbeidet til tidligere relevant arbeid.


\section{Future Work}
\subsection{Prediction algorithms}
There is a lot of potential future work to be done regarding social media and general recommendation. In the domain of movie recommendation the exploration of different recommendation and prediction algorithms is an interesting field. The most central regarding this work and algorithms is:
\begin{itemize}
    \item Runtime
    \item Score
    \item Runtime versus score (RMSE)
    \item Parallelization
    \item Scalability
\end{itemize}
Runtime and score was just touched during this research, but opened for some interesting future work to be done, with well established testing potential through RMSE when comparing different approaches.

Just when it comes to the prediction approach taken in this research different versions should be explored. Variants of k-NN can be used to select neighbors. Other interesting subjects for future work includes: How to handle outliers in a good way regarding the data, best approach to value neighbors, and exploring different heuristics for good $k$ values.


\subsection{Data modeling}
TODO
Outliers  normalizing dataset building


\subsection{Learners}
Comparing the actual ratings users are giving a movie with the predicted ratings, and having the system learn from these values. This would allow the system to not only depend on the harvested ratings and Netflix ratings, but get an extra source of prediction.

Future work in this field would include:
\begin{itemize}
    \item Exploring how to weigh a correct or faulty prediction
    \item If a long term learning system will be of any help
    \item Cost of having the system learn versus the score gained
    \item Potential scaling issues with learning
\end{itemize}


\subsection{Twitter harvesting}
TODO


\section{Summary}
TODO
The objective of this project was to demonstrate the need for scripting in a web- application, showing that it can add value to the user experience. We feel like we have accomplished this objective. We have shown what is possible to achieve with the technologies we had chosen, and what advantages they provided us with compared to more traditional technologies. We have demonstrated the flexibility of our system, which can be suited to almost any need and any domain without changing the core functionality. The system is in no way complete as a library system, but that wasn’t really a priority for this project. The rmivery we made ended up being a platform for developers with almost endless possibilities. We feel we have proved that the concept of a console in a web- application is useful, and that was the tasks at hand.

As far as we can see, there exists no similar systems in the world today. We feel we have created something new, something of great value. We have utilized new emerging technologies, with a lot of advantages over more traditional solutions. There is however still some work left to be done before a system based on our work can be used in a production environment. Our system presents a new way of thinking, in that it explicitly restricts functionality instead of explicitly adding functionality. It is way of thinking we think has a lot of potential, yet it requires more research on some of its aspects before we can draw definitive conclusions on its implications. Our product is not finished, but what we have created can be picked up by anyone. With more development and research we think it can be something that holds commercial business value.



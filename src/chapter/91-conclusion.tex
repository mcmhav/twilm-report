% !TEX root = ../report.tex

\chapter{Conclusion}

\minitoc

This chapter will describe the final version of our product and what we have found during this project. Also we will discuss the further development of the system and what improvement can be made.

\clearpage

\section{Final Product}
STATUS
TODO
The system we have created is a basic library system, and it allows users to store information on books and authors. It is a web application, meaning it is accessed through a web- browser. It separates itself from regular web- applications in that it incorporates a console. The system consists of two main components, a client part and a server with an associated database. The client part deals with user input while the server provides the clients the ability to persistently store data between user sessions.

\subsection{Chosen Technologies}
We spent a lot of time on the pre- study and on research in this project, and we feel it was well worth the effort. Also our customer helped us a lot, and offered great suggestions and guidance throughout the project. The technologies we ended up using were not only new and exciting, but were also a great fit together. JavaScript, jQuery, JSON, HTML and CSS are robust and mature web technologies in use all over the world today. Coupled with CouchDB, which was designed from the ground with web- applications in mind, they make a powerful collection of technologies that work well together.

Probably the most important decision on technology we were left with after the pre- study was whether to go for a traditional database management system like MySQL, or choose a more unconventional NoSQL system. The customer did not specify any requirements concerning the performance or stability of the server in this project, and indicated that the specifics on the server implementation was not that important to them. As a result we wanted it to be as simple as possible. After a detailed look at the needs for this project we ended up choosing a NoSQL database in the form of CouchDB. The reason for this was that CouchDB provided us with many advantages that a traditional relational database could not. Our system is heavily centered around JavaScript objects and their attributes. CouchDB, and other document oriented NoSQL systems, stores data in the form of objects directly in JSON format, which is a simple textual notation for JavaScript objects. This implied a minimal amount of work getting the client and server to communicate. If we had opted to use a traditional SQL system for the database we would have needed more extensive server side technologies, like PHP or ASP.NET to handle the database and to do database-object parsing. All of this was avoided by choosing CouchDB, which is able to work without any extra server software, and automatically creates a RESTful api to interact with the database.

Whereas trying to add undefined attributes to an existing record in a traditional relational database management system would result in an error, CouchDB automatically adds these attributes to the stored object. In essence, it allows us to add any new attribute to the existing objects, as long as it's represented in JSON format. This obviously gives you great flexibility. To do something similar in a regular SQL system you would have to change the schema of the database table every time a new attribute was discovered, or be able to morm every aspect of the objects from the beginning. For non-trivial real world domains, this usually presents a difficult challenge. Also the job of representing complex objects are considerably less than in a relational database. Complex object representation in SQL often requires a tedious process to identify the different tables you need and the relations between them. In CouchDB this job is reduced to just putting the entire JavaScript object directly into the database, which saves a lot of time for the developer. Maybe the best experience from this project was the freedom the technologies gave us. All the technologies were really simple out of the box, and this left us with the ability of choosing exactly which functionality to add. Also the technologies were flexible, meaning there were few limitations to what we could add. This flexibility means that is is easy both for developers and end users to add new functionality to the system. It should be mentioned that to be able to use this system to its maximum potential you need some sort of prior experience with JavaScript. But with this in hand you can accomplish virtually anything.


\section{Findings}
TODO
Oppsummer hva du har oppnådd,
sammenlignet med hva du opprinnelig ønsket å oppnå.

As discussed above, the system allows the user to dynamically define new object types and redefine existing ones, which is a rare, but powerful feature. In non-trivial domains it can be extremely difficult to create good a morm before the system is deployed, and the morm may need to change over time. Traditionally, database redesign would be a much more complicated and expensive process.
\cite{kroenke2006database}

As opposed to traditional systems where the developer has to explicitly add new functionality, our console solution features a complete feature set from the start, by means of the scripting language. The job of the developer is then to explicitly restrict unsafe functionality, offer simplifications to the syntax, and create alternative input methods(such as a rich GUI). A key advantage of implementing a console is the drastically reduced development cost of input commands as opposed to design and development of complex GUI elements. Exposing these to the user also poses a problem; Another important advantage of a hybrid console/GUI system is that the graphical part of the application can be simpler and therefore more likely to be user friendly, since the more complex functionality which is only used by power users can be accessed through the console.

A backside to the flexibility of the scripting console is that it poses a challenge when it comes to restricting unwanted functionality in the system. This was not something we focused on in this project, as our efforts were ultimately focused on finding the potential of the console. Our prototype assumes that the users know what they are doing and have no desire to cause problems in the system. However, we believe that this can be solved on a case-by-case basis if the solution is to be developed further. Another potential issue related to this would be security. In cases where this is a concern, ensuring that the user does not perform malicious actions would be of outmost importance. Security was not a point of focus for our project; However, in our solution, all commands are executed on the client side, and only JSON objects are sent to the server. We believe that in most domains, it is viable to implement a more complex server that tests the received objects for consistency to eliminate activity that is harmful to the system.


\section{Related Work}
TODO
Relater arbeidet til tidligere relevant arbeid.


\section{Future Work}

\subsection{Prediction algorithms}
There is a lot of potential future work to be done regarding social media and general recommendation. In the domain of movie recommendation the exploration of different recommendation and prediction algorithms is an interesting field. The most central regarding this work and algorithms is:
\begin{itemize}
    \item Runtime
    \item Score
    \item Runtime versus score (RMSE)
    \item Parallelization
    \item Scalability
\end{itemize}
Runtime and score was just touched during this research, but opened for some interesting future work to be done, with well established testing potential through RMSE when comparing different approaches.

Just when it comes to the prediction approach taken in this research different versions should be explored. Variants of k-NN can be used to select neighbors. Other interesting subjects for future work includes: How to handle outliers in a good way regarding the data, best approach to value neighbors, and exploring different heuristics for good $k$ values.


\subsection{Data modeling}
TODO
Outliers  normalizing dataset building


\subsection{Learners}
Comparing the actual ratings users are giving a movie with the predicted ratings, and having the system learn from these values. This would allow the system to not only depend on the harvested ratings and Netflix ratings, but get an extra source of prediction.

Future work in this field would include:
\begin{itemize}
    \item Exploring how to weigh a correct or faulty prediction
    \item If a long term learning system will be of any help
    \item Cost of having the system learn versus the score gained
    \item Potential scaling issues with learning
\end{itemize}


\subsection{Twitter harvesting}
TODO


\section{Summary}
TODO
The objective of this project was to demonstrate the need for scripting in a web- application, showing that it can add value to the user experience. We feel like we have accomplished this objective. We have shown what is possible to achieve with the technologies we had chosen, and what advantages they provided us with compared to more traditional technologies. We have demonstrated the flexibility of our system, which can be suited to almost any need and any domain without changing the core functionality. The system is in no way complete as a library system, but that wasn’t really a priority for this project. The rmivery we made ended up being a platform for developers with almost endless possibilities. We feel we have proved that the concept of a console in a web- application is useful, and that was the tasks at hand.

As far as we can see, there exists no similar systems in the world today. We feel we have created something new, something of great value. We have utilized new emerging technologies, with a lot of advantages over more traditional solutions. There is however still some work left to be done before a system based on our work can be used in a production environment. Our system presents a new way of thinking, in that it explicitly restricts functionality instead of explicitly adding functionality. It is way of thinking we think has a lot of potential, yet it requires more research on some of its aspects before we can draw definitive conclusions on its implications. Our product is not finished, but what we have created can be picked up by anyone. With more development and research we think it can be something that holds commercial business value.



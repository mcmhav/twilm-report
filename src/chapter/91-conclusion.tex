% !TEX root = ../report.tex

\chapter{Conclusion}

\minitoc

This chapter will describe the final version of our product and what we have found during this project. Also we will discuss the further development of the system and what improvement can be made.

\clearpage

\section{Final Product}
STATUS - beskriv status for ditt arbeid
TODO
We spent a lot of time on the preliminary study and doing research to build the ground for our work. This was to decide which direction to take amongst the vast amount possibilities. We see this as well spent time. Similar solutions~\ref{sec:similarsys} was of great help, and let us understand how different approaches had been done by previous systems and what worked from them.

\subsection{The harvestere}
TODO Something about amanda or whatevah - the scraper

\subsection{The database}
The choice of database was a good match to the data harvested from Twitter. MongoDB is a easily scalable database, with easy replication possibilities. The data from Twitter consists of different kinds of field, and would therefore have been more troublesome to work with in a SQL database. MongoDB, as other document oriented NoSQL databases, stores the data as JavaScript Object Notation, which is the same notation as fetched from Twitter ? TODO, chech this fact lol. And there would therefore not be needed any parsing from the Twitter harvesting to the database.

\subsection{The predictor}
The prediction part was never implemented TODO: refer to something about why. But research around good prediction candidates has been made, and is and interesting topic for future work.


\section{Findings}
It was interesting to see from the research done how many different approaches there are to take when suggesting user ratings, and the importance of diversity when calculating the predictions is key. There is no one model when suggesting movies to users, which will be perfect for all. Even the top two winners of the Netflix-prize competition scored better when their solutions were merged together. Even though the winners managed to produce a 10\% better score than the Netflix's own system, the winners solution was not taken into production. The algorithms used to recommend at the winners level would be too time consuming to use on the complete data, and would not work well with the constant updating values of the actual Netflix users ratings, whereas the Netflix-prize competitors where working on a unknown but stationary set of users and ratings.

TODO - Twitter

\subsection{Goals reached}
TODO - discuss the goals...
The goals reached was

\subsection{Goals unreached}
sammenlignet med hva du opprinnelig ønsket å oppnå.


\section{Related Work}
Relater arbeidet til tidligere relevant arbeid.
As we saw from the preliminary study on similar systems~\ref{sec:similarsys}, there are some systems with similar characteristics as the approach taken in this project. They were of great help when exploring how to attack the issues that comes along with recommending movies, gathering information from social media and exploring big amounts of data.


\section{Future Work}
\subsection{Prediction algorithms}
There is a lot of potential future work to be done regarding social media and general recommendation. In the domain of movie recommendation the exploration of different recommendation and prediction algorithms is an interesting field. The most central regarding this work and algorithms is:
\begin{itemize}
    \item Runtime
    \item Score
    \item Runtime versus score (RMSE)
    \item Parallelization
    \item Scalability
\end{itemize}
Runtime and score was just touched during this research, but opened for some interesting future work to be done, with well established testing potential through RMSE when comparing different approaches.

Just when it comes to the prediction approach taken in this research different versions should be explored. Variants of k-NN can be used to select neighbors. Other interesting subjects for future work includes: How to handle outliers in a good way regarding the data, best approach to value neighbors, and exploring different heuristics for good $k$ values.


\subsection{Data modeling}
TODO
Outliers  normalizing dataset building


\subsection{Learners}
Comparing the actual ratings users are giving a movie with the predicted ratings, and having the system learn from these values. This would allow the system to not only depend on the harvested ratings and Netflix ratings, but get an extra source of prediction.

Future work in this field would include:
\begin{itemize}
    \item Exploring how to weigh a correct or faulty prediction
    \item If a long term learning system will be of any help
    \item Cost of having the system learn versus the score gained
    \item Potential scaling issues with learning
\end{itemize}


\subsection{Twitter harvesting}
TODO


\section{Summary}
TODO
The objective of this project was to demonstrate the need for scripting in a web- application, showing that it can add value to the user experience. We feel like we have accomplished this objective. We have shown what is possible to achieve with the technologies we had chosen, and what advantages they provided us with compared to more traditional technologies. We have demonstrated the flexibility of our system, which can be suited to almost any need and any domain without changing the core functionality. The system is in no way complete as a library system, but that wasn’t really a priority for this project. The rmivery we made ended up being a platform for developers with almost endless possibilities. We feel we have proved that the concept of a console in a web- application is useful, and that was the tasks at hand.

As far as we can see, there exists no similar systems in the world today. We feel we have created something new, something of great value. We have utilized new emerging technologies, with a lot of advantages over more traditional solutions. There is however still some work left to be done before a system based on our work can be used in a production environment. Our system presents a new way of thinking, in that it explicitly restricts functionality instead of explicitly adding functionality. It is way of thinking we think has a lot of potential, yet it requires more research on some of its aspects before we can draw definitive conclusions on its implications. Our product is not finished, but what we have created can be picked up by anyone. With more development and research we think it can be something that holds commercial business value.



% !TEX root = ../report.tex

\chapter{Evaluation}

\minitoc

I evalueringskapittelet beskriv hvordan du vurderer
arbeidet ditt. Oppsummer evalueringsresultatene, og
bruk dem til å vurdere ditt eget arbeid kritisk. Vær
ærlig om eventuelle mangler.
Hva betyr resultatene? (Signifikans)

This is the final phase of this project, the evaluation. Here it will be discussed why and how the outcome ended up being what it is today. This includes how the team worked together, why it gave the result it did, the cooperation with the customer, and how it was working with an overseeing force. Furthermore, we will discuss the issues met during the project, and how the process we used worked for us.

\clearpage


\section{Development Process}
This section will explain the usage of the chosen development methodology, and the good and bad parts with using this methodology.

\subsection*{Good}
As mentioned above, Scrum allowed us be flexible and made it easy to change the direction the system was taking. This would have been troublesome without an agile development methodology. Traditional methods such as the Waterfall morm are rigid and designed to include only one design phase, one implementation phase, etc. This would have stopped us from being able to handle the redirection we did between sprint two and three towards a more dynamic system. Scrum let us to scale down in each sprint, focusing on smaller tasks. It also made the involvement of all the members in each aspect of the project easy, since Scrum opened a forum where we shared with each other.

Our execution of the Scrum process was far from perfect, as was to be expected since none of us had any prior experience in using it. But we learned as we went, and feel the experiences we have accumulated in this project will help us in the years to come. Scrum is the development process of choice in most projects concerning IT and we are bound to face this development morm in the future. This project provided us with first hand experience in the importance of constantly following up the customer to be able to rmiver a product that meets their expectations.

\subsection*{Bad}
The Scrum process imposes a lot of rules, activities and meetings, which was time consuming. The overhead produced by each meeting, each demo presentation, etc, could in some cases have been better spent towards other activities instead. Without all this overhead we would have gotten more time to focus on the implementation, and maybe include more features in the product.


\section{Testing}
We mainly performed three types of testing in this project, unit testing of the code, test cases that were made for each user story and acceptance testing on each user story with the customer. The test cases were performed towards the end of each sprint, and in multiple cases helped us discover minor bugs or shortcomings of the system. We feel like this gave us sufficient test coverage of the different functionality and parts of the system. According to the test plan we were also supposed to do user and system testing, however this was not done. The reason for this is explained below. All in all we are satisfied with our testing process and the amount of testing we were able to do.

We originally planned to include user testing in this project. In the beginning the project was user- centered focusing on the user experience and usability of the system. It was important for us to involve users in the development process. However as the project developed, it became increasingly technology- centered. The priority shifted to discovering the true potential of the technologies we had chosen. As a result of this change in priorities and the limited time available, user testing was not performed as a part of this project. It is however something we would have liked to do if we were given more time, to test different users response to our system and identify improvements that can be made.

Our system turned out to be something else than what we expected from the beginning. The product in its current state is more a platform for developers than a finished product to be set out in production. Because of this it was difficult to identify what to look for in a complete system test of the final product, we did not posses an extensive list of requirements we could base it on. The different components of the system had already been extensively tested at the end of each sprint, both through test cases for each user story and unit tests of the code. Also it would have been difficult to test for specific functionality as the product is able to do pretty much anything. Although it is a library system it does not make sense to test it as a one, as it was never the intended goal for this project to produce a functioning library system. Taking all of this into account we decided not to perform a complete system test on the final product.


\section{Issues}

\subsection*{Group size}
The biggest issue the team encountered was that the team size ended at four. The intended group size for the project was five to seven, which should have produced an extra 325 - 975 work hours, which is a considerable amount of hours. We managed to come out on top of this situation because of a set of responses and effects of having a small group.

\subsubsection*{Ease of communication}
Since we were of such a small size the communication was tight and keeping everyone updated was easy to achieve. This made production efficiency high since there was not done any double work.

\subsubsection*{The team members}
Taking responsibility came more naturally with such a small group as ours. The members stepped up their game and rmivered when it was needed. The general goal of the team was to rmiver a good result. This together with a great chemistry between the members made producing a good result with a small group size achievable.

\subsubsection*{Risk handling}
Since we were dealing with a prototype project, we always had in mind that requirements could change, and therefore had ease of modifiability in the back of our mind, while developing the system.

\section{Summary}
The course has all in all proved to be a positive and valuable experience for us all. For most of us, the experience of working on such a big project in a team was quite new. We experienced how important it was to plan ahead, to distribute the workload, and to collaborate to achieve a common goal. This was also our first experience in working with an external customer, giving us valuable experiences in this type of project. These experiences are sure to prove useful for in the years to come when we participate in similar projects.

We, as a group, feel that we have reached our goal, and rmivered a great product that the customer was very satisfied with. We made our customer happy and exceeded his expectations, and looking back this achievement is something we can be proud of.
